\documentclass{agstract}

% ================================================================== %
%                                                                    %
%                              Document                              %
%                                                                    %
% ================================================================== %

% ----------------------- Header formatting ------------------------ %

\name{Forest Kobayashi}
\orderedNumber{14}
\assignment{9w}
\dueday{Wednesday} % accepts Monday, monday, Wednesday, wednesday, etc
\duedate{3/21/18}
\reading{11.3}
\problems{10.2.17, 10.3.9, \emph{10.3.11}, 11.1.2, 11.1.10}
\acknowledgements{{Cole K., Owen G.}, {}, {}, {kfldsafjksd}, {}}
\onTime{0} % Takes 0, 1, 2, 3, 4, 5, 6

\lfoot{Due Wednesday, March 21nd}

\begin{document}

% --------------------------- Problem 1 ---------------------------- %

  \section*{Problem 10.2.17 (The alternating group of degree 5)}
    Use problems \textbf{6.2.15} and \textbf{\emph{10.2.15}} and
    Lagrange's theorem to prove that $A_5$, the alternating group of
    degree 5, has no non-trivial normal subgroups.

  \hrulefill

  \section*{Solution:}

  \clearpage

% --------------------------- Problem 2 ---------------------------- %

  \section*{Problem 10.3.9}
    Let $G$ be a group, and let $N \triangleleft G$. Assume that
    $\abs{G : N} = m$. Let $x \in G$. Prove that $x^m \in N$.

  \hrulefill

  \section*{Solution:}

  \clearpage

% --------------------------- Problem 3 ---------------------------- %

  \section*{Problem \emph{10.3.11}}
    Assume that $N$ is a normal subgroup of a group $G$. Assume $E$ is
    a subgroup of $G/N$. Thus $E$ is a collection of right cosets of
    $N$ in $G$. Let $K$ be the union of all the elements of $E$. In
    other words, $K$ is a subset of $G$ consisting of all the elements
    in the right cosets in $E$. Prove that $K$ is a subgroup of $G$
    that contains $N$. What is $\abs{K}$?

  \hrulefill

  \section*{Solution:}

  \clearpage

% --------------------------- Problem 4 ---------------------------- %

  \section*{Problem 11.1.2}
    Define $\phi : \zpmod{8} \to \zpmod{8}$ by $\phi(x) = 2x$. Is
    $\phi$ a homomorphism? If so, what is $\phi^{-1}(\set{0})$? Answer
    the same questions for $\theta : \zpmod{8} \to \zpmod{8}$ defined
    by $\theta(x) = x^2$.

  \hrulefill

  \section*{Solution:}

  \clearpage

% --------------------------- Problem 5 ---------------------------- %

  \section*{Problem 11.1.10}
    Let $\phi : G \to H$ be an onto homomorphism.
    \begin{enumerate}
      \item Assume that $G$ is abelian. Does this imply that $H$ is
        abelian? What about the converse?
      \item What if we replaced abelian by cyclic in the above
        question?
    \end{enumerate}

  \hrulefill

  \section*{Solution:}

\end{document}
